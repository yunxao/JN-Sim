\chapter{Apendice: Uso del simulador}

\section{Ejecuci�n del simulador}

Para ejecutar el simulador, basta con ejecutar el archivo ejecutable.

Si se desea ejecutar desde el c�digo, la clase principal donde se encuentra la
consola del simulador es \cod{drcl.ruv.System}. 

Opcionalmente existe una clase que te permite generar ficheros base para crear
simulaciones. Esta clase se encuentra en \cod{tid.util.XMLGenerator}

\section{Ejecuci�n de simulaciones}

Una vez arrancado el simulador se pueden cargar las simulaciones de los
ficheros XML. Para hacerlo es necesario lanzar un script \ac{TCL}. El script
\cod{baseScript.tcl} te solicita los ficheros de topolog�a y eventos. 

\section{Estructura de los ficheros XML\labelSec{estructuraXML}}

La simulaci�n consta de dos ficheros XML que almacenan lo necesario para la
simulaci�n. El fichero de la topolog�a almacena la
configuraci�n de la red, los nodos y los protocolos usados. El fichero de
eventos almacena la duraci�n de la simulaci�n y que ocurre durante la misma.

El formato de los ficheros XML se pueden consultar en la web

\begin{itemize}
 \item Topolog�a: http://www.xp-dev.com/wiki/79623/TopologyXML
 \item Eventos: http://www.xp-dev.com/wiki/79623/EventsXML
\end{itemize}
