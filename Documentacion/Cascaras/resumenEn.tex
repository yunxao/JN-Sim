%---------------------------------------------------------------------
%
%                      resumen.tex
%
%---------------------------------------------------------------------
%
% Contiene el cap�tulo del resumen.
%
% Se crea como un cap�tulo sin numeraci�n.
%
%---------------------------------------------------------------------


\chapter{Abstract}
\cabeceraEspecial{Abstract}

The Border Gateway Protocol (BGP-4)\cite{rfc4271} is an inter-Autonomous System
routing protocol.

The primary function of a \ac{BGP} speaking system is to exchange network   
reachability information with other \ac{BGP} systems.  This network reachability
information includes information on the list of Autonomous Systems that
reachability information traverses. This information is sufficient for
constructing a graph of AS connectivity for this reachability, from which
routing loops may be pruned and, at the AS level, some policy decisions may
be enforced.

\acs{BGP} is the protocol backing the core routing decisions on the Internet
and, since most Internet service providers must use \ac{BGP} to establish
routing between one another, it is one of the most important protocols of the
Internet.

Nowadays, the study of \ac{BGP} is very limited because the tools available to
analyze the protocol are limited and have scalability issues, are not fully
implemented or do not allow to go in depth into the internal behavior of the
protocol because they are proprietary software.

The objectives of this project are:
\begin{itemize}
 \item Research and improve the tools available for studying \ac{BGP}.
 \item Study the \ac{BGP} protocol with the researched tools.
\end{itemize}

~\vfill
{\bf Keywords:} \parbox[t]{.75\textwidth}{\keywords, \keywordsEn}

\endinput
% Variable local para emacs, para  que encuentre el fichero maestro de
% compilaci�n y funcionen mejor algunas teclas r�pidas de AucTeX
%%%
%%% Local Variables:
%%% mode: latex
%%% TeX-master: "../Tesis.tex"
%%% End:
