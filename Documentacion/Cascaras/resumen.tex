%---------------------------------------------------------------------
%
%                      resumen.tex
%
%---------------------------------------------------------------------
%
% Contiene el cap�tulo del resumen.
%
% Se crea como un cap�tulo sin numeraci�n.
%
%---------------------------------------------------------------------
 
\chapter{Resumen}
\cabeceraEspecial{Resumen}

El \ac{BGP}\cite{rfc4271} es un protocolo de encaminamiento entre sistemas
aut�nomos. 

La funci�n principal del sistema se comunica con \ac{BGP} es el intercambio de
informaci�n de accesibilidad de la red con otros sistemas \ac{BGP}. Esta
informaci�n de accesibilidad incluye informaci�n sobre la lista de Sistemas
Aut�nomos que la informaci�n ha atravesado. Esta informaci�n es suficiente para
construir el grafo que de acceso a la red, podar bucles de encaminamiento y
aplicar pol�ticas en el proceso de decisi�n a nivel de sistema aut�nomo. 

\ac{BGP} es el protocolo de apoyo del proceso de decisi�n de encaminamiento de
Internet y, como la mayor�a de los proveedores de servicios de Internet deben
utilizar \ac{BGP} para establecer rutas entre unos y otros, es uno de los
protocolos m�s importantes de Internet.

Actualmente el estudio de \ac{BGP} est� muy limitado debido a que las
herramientas que existen para analizar el protocolo son escasas y tienen
problemas de escalabilidad, no est�n completamente implementadas o no permiten
profundizar m�s en el comportamiento interno del protocolo debido a que son de
software propietario.

Los objetivos de este proyecto son:

\begin{itemize}
 \item Investigar y mejorar las herramientas disponibles para el estudio de
\ac{BGP}.
 \item Estudiar \ac{BGP} con las herramientas investigadas.
%  \item Crear y verificar herramientas de protocolos nuevos
\end{itemize}
~\vfill
{\bf Palabras Claves:} \parbox[t]{.75\textwidth}{\keywords, \keywordsEs}

\endinput
% Variable local para emacs, para  que encuentre el fichero maestro de
% compilaci�n y funcionen mejor algunas teclas r�pidas de AucTeX
%%%
%%% Local Variables:
%%% mode: latex
%%% TeX-master: "../Tesis.tex"
%%% End:
