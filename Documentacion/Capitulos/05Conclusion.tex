\chapter{Conclusi�n}

\section{Simulador}

El protocolo \ac{BGP} es uno de los pilares que hace que Internet funcione
correctamente. Su red abarca gran cantidad de nodos que comparten su
informaci�n con el resto de nodos de la red.

Sin embargo, pese a estar muy extendido, existen pocas herramientas que
permitan estudiar el protocolo y las alternativas que existen son software
propietario o son poco escalables.

Uno de los objetivos del proyecto ha consistido en crear un entorno de
simulaci�n completo, escalable y de f�cil uso para \ac{BGP}. Se ha integrando
el lenguaje de marcas XML que permite usar herramientas para crear y
automatizar la creaci�n de topolog�as. 

Tambi�n se ha creado un sistema de trazas que facilitan el an�lisis del
protocolo despu�s de la simulaci�n.

El simulador se ha hecho m�s modular, lo que facilita la implementaci�n
de nuevos protocolos, para su validaci�n, y herramientas para el simulador. La
simulaci�n de estos se benefician del uso del XML al igual que \ac{BGP}.

El resultado ha sido un simulador f�cil de usar y escalable. No se sacrifica la
potencia del simulador y se mantiene la integraci�n con el lenguaje de script
\ac{TCL} lo que permite crear simulaciones m�s complejas. Gracias a la
modularizaci�n del simulacor se pueden implementar nuevos protocolos y
herramientas que usen las nuevas caracter�sticas y se beneficien de la potencia
del mismo.

\section{Estudio \ac{BGP}}

La segunda parte ha consistido en, una vez desarrolladas las herramientas
necesarias para el estudio de \ac{BGP}, utilizarlas para comprender mejor el
protocolo. 

Con los conocimientos adquiridos en el proyecto, el estudio se ha centrado
en el proceso de decisi�n, que es el punto cr�tico de \ac{BGP} y donde se 
registran la mayor cantidad de problemas.

Se ha observado que los principales puntos conflictivos del proceso de decisi�n
son el uso de pol�ticas y atributos para calcular los pesos de las rutas de cara
a su publicaci�n a los nodos vecinos. 

El uso de estas caracter�stica causa efectos no deseable son la aparici�n de
oscilaciones y \angl{wedgies}. Estos fen�menos dif�ciles de prever en redes
complejas. 

Cuando redes de gran tama�o, como Internet, sufren oscilaciones y
\angl{wedgies}, los problemas que �stas acarrean pueden derivar en p�rdidas
econ�micas en empresas del sector de la telecomunicaci�n. 


