\chapter{Lista de abreviaturas, acr�nimos y definiciones}
\begin{acronym}
 \acro{ACA}		{Arquitectura de componentes aut�nomos}
 \acro{ASPATH}	{AS\_PATH}	{Autonomous System Path}
 \acro{AS}		{Sistema Aut�nomo} %X
 \acrodefplural{AS}	[AS]	{Sistemas Aut�nomos} 
 \acro{BGP}	[BGP-4]	{Border Gateway Protocol} %ref4271
 \acro{BSD}		{Berkeley Software Distribution} %X
 \acro{CSL}		{Capa de Servicios de Red}
 \acro{DML}		{Domain Modeling Language}
 \acro{EBGP}	[eBGP]	{BGP externo} %X
 \acro{EGP}		{Protocolo de encaminamiento externo} %X
 \acro{IBGP}	[iBGP]	{BGP interno} %X
 \acro{IGP}		{Protocolo de enrutamiento interno} %X
 \acro{ISP}		{Proveedor de Internet} %X
 \acrodefplural{ISP}	[ISP's]	{Proveedores de Internet}
 \acro{JSIM}	[J-Sim]	{JavaSim} %X
 \acro{MED}		{Multi-Exit Discriminator}
 \acro{NLRI}		{Network Layer Reachability Information}
 \acro{OSPF}		{Open Shortest Path First} %X
 \acro{RIP}		{Routing Information Protocol} %X
 \acro{RIPE}		{R�seaux IP Europ�ens} %X	
 \acro{RIR}		{Registro Regional de Internet}
 \acro{RFC}		{Request For Comments} %X
 \acro{RR}		{Reflector de Rutas} %X
 \acro{RRC}		{Cliente de reflexi�n de rutas} %X
 \acro{TCL}	[Tcl]	{Tool Command Language}
 \acro{TK}	[Tk]	{Tool Kit}
 \acro{XML}		{Extensible Markup Language}
\end{acronym}
